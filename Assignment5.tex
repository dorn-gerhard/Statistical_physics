\documentclass[12pt,a4paper]{article} %scrbook}%bookof}



% makes nice font


%fills A4 page
\usepackage{a4wide}
\usepackage[inner=2cm,outer=2cm, bottom = 2.0cm]{geometry}   %nice setting for twoside binding



%\usepackage[utf8]{inputenc}   % damit man Umlaute direkt eingeben kann und diese erkannt werden.
\usepackage[T1]{fontenc}        % Umlaute werden als eine Einheit angesehen -> richtige Trennung;
% ausserdem: die T1-Fonts gibt es auch in
% größeren Größen
%\usepackage{ae,aecompl}
\usepackage{units}
\usepackage[english]{babel}      %hiermit erhält man z.B. 'Abb.' statt 'Fig.' bei Bildbeschriftungen
\usepackage[utf8]{inputenc}
%\makeglossary                   % zwischendurch makeindex disseration aufrufen
\usepackage{enumerate}
\usepackage{latexsym}


%\clearscrplain
%\clearscrheadings

\usepackage[headsepline]{scrlayer-scrpage}
%\usepackage[]{scrpage2}
%\clearplainofpairofpagestyles
%\clearscrheadings
\pagestyle{scrheadings}

\cofoot[\pagemark]{\pagemark}% optionales Argument --> plain
%\clearscrheadings
\rohead{NAWI Graz}
\chead{
	Exercise Statistical Physics \\{\small Contact: gerhard.dorn@tugraz.at}}
\lohead{ITP}
\usepackage{titlesec}
\titleformat{\section}
{\normalfont\large\bfseries}{\thesection}{1em}{}


\usepackage{amsmath,amsthm,amsfonts}
%\usepackage{showkeys}
%\usepackage{epsfig}
%\usepackage{epic}
\usepackage{xr}

\usepackage{graphicx}
%\usepackage{inmthesis}
\usepackage{hyperref}
\usepackage{verbatim}
\usepackage{tikz}
\usepackage{enumitem}

\usepackage[sorting=none, hyperref=true, backref=true]{biblatex}
\addbibresource{./library.bib}
%boxes
\usepackage{mdframed}
\usepackage{tensor}
%unity operator
%\usepackage{bbold}
 \setcounter{section}{12}
\begin{document}

 %\begin{center}
 %Technische Universität Graz \\
 %Institut für Theoretische Physik - computational physics \\
 %Prof. Dr. Wolfgang von der Linden \\
 %Kontakt: gerhard.dorn@tugraz.at
 %\end{center}
 
 
 \vspace{1cm}
 
 \section{A two--level system: prototype of a heat storage}
 \paragraph{Aim:} This exercise serves you to repeat the basic concepts of statistical physics, namely how to derive thermodynamic properties from a canonical partition sum. Basic conepts such as in/distinguishability, non--interacting particles, specific heat (heat storage) and pressure, will be discussed. Some mathematical knowledge about series is required!
 \paragraph{Time:} 2-3 hours (if the knowledge of the lecture is there!)\\
 
Given is a system of volume $V$ of $N$ not interacting particles, which can be in two energy states $\epsilon_1 = 0$ and $\epsilon_2 = \epsilon > 0$. Exchange of energy with a heat bath is possible.
Our aim is to derive thermodynamic properties via the partition sum $Z(T,N,V)$ for distinguishable (\textbf{I}) and indistinguishable (\textbf{II}) particles. We start with distinguishable particles. First we have to ask the question how to count states. 
\begin{enumerate}

 \item What is the partition sum for one particle ($N = 1$)?
 \item Calculate for a fixed number of excited states $n$~\footnote{Subset $n$ of $N$ particles, which are in the excited state (state 2)} the Hamilton function (energy). 
 \item How many configurations with exactly this energy are there for distinguishable and indistinguishable particles?
 
 \item Now calculate the total partition sum for distinguishable (\textbf{I}) without restriction of the number of excited states, only the total particle number $N$ is fixed. \\Hint: binomial formula $(a+b)^N$.
 \item How does the partition sum of indistinguishable particles look like? Calculate it. \\Hint: Geometric series. 
 \end{enumerate}
 The following steps have to be performed for distinguishable (\textbf{I}) and indistinguishable (\textbf{II}) particles.
 \begin{enumerate}[resume]
 \item What is the probability to find $n$ particles in the excited energy level?
 \item What is the average particle number in the excited energy level? Sketch this quantity depending on the temperature $T$.\\Hint (\textbf{I}): Write out the binomial coefficient and rewrite it.\\
 Hint (\textbf{II}): Use a derivative.
 
 \item What is the average internal energy $U$ in the system?\footnote{Use the fact that we have a non--interacting system.} What can we say about energy fluctuations in this system? (look into the lecture notes) Quantify them.
 \item Calculate the thermodynamic potential, derive the entropy $S$ and make the connection to the internal energy $U$.
 
 \item Calculate and sketch the specific heat depending on the the temperature $T$. Hint: hyperbolic functions
 \item When is this system best suited as heat storage? \\ 
 (You may search a local maximum (if it exists), simplify the terms as far as possible, solve numerically or graphically)\\
 In which case (\textbf{I} or \textbf{II}) can we have a higher specific heat?
 
  
 \item Now assume that the higher energy level depends from the volume:
 $$ \epsilon = \frac{1}{V}.$$
 \item Calculate the pressure of the system depending on the temperature and the volume.\\
 Try to establish a connection between pressure and internal energy.
 
\end{enumerate}
 

 
 \section{Einstein solid}
 \paragraph{Aim:} Application of statistical tools to derive the specific heat of crystals. You may need differentiation rules for hyperbolic functions
 \paragraph{Time:} 1 hour (after having solved 13 this should be easy) \\
 
 The Einstein solid is a rather simple model for a solid based on the following assumptions (we consider a total of $N$ atoms in contact with a thermal bath):
 \begin{itemize}
  \item Each of the atoms has the same mass $m$
  \item Each of the atoms in the solid is an independent 3D quantum harmonic oscillator 
  \item All atoms oscillate with the same frequency\footnote{This constraint is simple but not always correct. The Debye model represents a more accurate description of a solid with variable frequencies.}
 \end{itemize}

 The Hamiltonian of a 3D harmonic oscillator is (we have a total of $3N$ degrees of freedom):
 \begin{align}
  H = \sum_{i = 1}^{3N} \frac{1}{2} m (\dot{q}_i^2 + \omega^2 q_i^2)
 \end{align}
\begin{enumerate}
 \item Remember from your quantum mechanics course: What are the eigenstates $\varepsilon_n$ of such a single quantum mechanical harmonic oscillator? 
 \item Calculate the partition sum for one oscillator. You should end up with some hyperbolic function. (assume an infinite number of possible eigenstates).
 \item Calculate the partition sum for $N$ particles. \\
 Hint: You may use the fact, that the oscillators are independent. $Z(N,T) = Z(1,T)^N$
 \item Calculate the thermodynamic potential
 \item Calculate the entropy
 \item Calculate the internal energy
 \item Calculate the specific heat
 \item How does the specific heat behaves in the low and high temperature limit?\\
 Which behaviour is in agreement with the physical laws (name them, what do they say)?
 \item[+] (half a bonus point) Proof that a 3D harmonic oscillator is equivalent to 3 1D oscillators by working with quantizations in 3 dimensions $n = n_x + n_y + n_z$. Find out the degeneracy for a fixed  energy characterized by the integer quantum number $n$ and calculate the partition sum. Show that this is equivalent to $Z(1,T)^3$ in one dimensions.
\end{enumerate}

\section*{Interludium: Non--interacting systems}

 The following points can help you to better understand the concept of non--interacting particles. You are not supposed to prepare those tasks for the exercise.
 \begin{enumerate}
 \item Which property of a Hamilton operator must be given to be able to speek from a interaction--free system? State an Hamilton which is interacting and proof that this property is violated.
 \item Proof that the canonical partition sum of a non--interacting (bosonic or fermionic) system can be expressed via the partition sum of a single particle (degree of freedom). 
 
 To do so start with the sum over all quantum numbers $\{n_1, n_2, \dots, n_N\}$\footnote{The energies are given as functions of those quantum numbers $n_j$: $E_j = f(n_j)$.}, split the Hamilton function according to those quantum numbers and conclude the factorizability.
 
 \item Since the summands in the (grand)canonical ensemble depend only on the energy of the states we look at another important transformation:\\
 \textbf{Energy representation:}
 We do not sum over states but over energies and take into account how many states with the same energy occur (degeneracy). This number of states per energy is called \textbf{density of states}. Perform this step formally also for interacting systems.
 
  \item Derive the mean occupation of the $k^{\textnormal{th}}$ fermionic or bosonic\footnote{Recall: Each fermionic energy level can only be occupied once, each bosonic energy level can be occupied arbitrarily.} energy state. Use the eigenenergies $\epsilon_i$ and the occupation numbers $n_i$: $H = \sum_i \epsilon_i n_i$.
  \end{enumerate}


 \section{Debye Modell}
 \paragraph{Aim:} In this problem we improof the Einstein model by taking different oscillation frequencies into account. 
 \paragraph{Time:} 0.5 - 1 hour
 
The harmonic oscillator of $N$ independent (not interacting) quantum mechanical particles is given by the following Hamilton operator:
\begin{align*}
 H = \sum_{j = 1}^{3 N} \Big[ \frac{p_j^2}{2m} + \frac{m \omega_j^2}{2} r_j^2\Big],
\end{align*} with the eigenenergies
$$E_j = \hbar \omega_j (n_j  + \frac{1}{2})  , \qquad n_{j} = 0,1,2,\dots 
$$

The energies of each degree of freedom $j$ (in total $3N$ degrees of freedom) is now only dependent from the oscillation frequency  $\omega_j$ and the excitation (quantum state) $n_j$ (in contrast to the previously discussed Einstein model where all oscillators had the same frequency).

\begin{enumerate}
  \item Write down the partition sum for fixed $\omega_j$, and arbitrary quantum numbers $n_j = 1,2,3,\dots$ in der canonical ensemble.
  Derive a partition sum that depends on $\omega_j$ as product over all $j = 1, \dots, N$ (Perform the summatrion over all $n_j$).\\
  Hint: Geometric series; try to express the final expression with hyperbolic functions.
  
  Derive the free energy $F$ and the internal energy $U$. You should end up with 
\begin{align}
U(T,N) = \sum_j \frac{\hbar \omega_j}{2} \coth{\frac{\beta \hbar \omega_j}{2}}. \label{equ:internal_energy} 
\end{align}

  
  \end{enumerate}
  What we need now is a distribution of the angular frequencies $D(\omega)$ (density of states) to replace the sum by an integral.
  The following bullets indicate how the Debye model can be derived (just for information):
 
  \begin{itemize}
   \item The number of oscillations with a given an angular frequency $\omega$ scales in 3D like $\omega^2$. Therefore the assumed density of angular frequency has the form $$D(\omega) = C \omega^2.$$
   
   \item We assume that there is a maximal frequency, the Debye frequency $\omega_D$ up to which the density is nonzero.
   $$ D(\omega) =  \begin{cases} C \omega^2 & \omega \leq \omega_D \\
                    0 & \omega > \omega_D
                   \end{cases}$$
    \item Normalization: The integral of the density of angular momentum should be equal to the the number of degrees of freedom since it will be used in a sum over those $\sum_{j=1}^{3N}$.

    \begin{align*}
    \int_0^{\omega_D} D(\omega) d\omega = \frac{C\omega_D^3}{3} \overset{!}{=}  3N \\
    C = \frac{9N}{\omega_D^3}
    \end{align*}
    \item Using the linear isotropic disperions relation $\boldsymbol{k} = c_S \boldsymbol{\omega}$ with the speed of sound in the crystal $c_S$ we can relate the angular frequency $\omega$ to the oscillations possible in the crystal. 
   \item In order to count the number of oscillations in the crystal, we characterize an oscillation (the mode) by its wave vector $\boldsymbol{k}$ and assume that those are distributed equally in the reciprocal space\footnote{Each grid point corresponds to one oscillation.}. 
   \item Given the dimensions of the crystal by $V = L \cdot L \cdot L$ we can estimate the minimal possible wave length $\lambda$ and wave vector $k = \frac{2\pi}{\lambda}$ by $k_{\textnormal{min}} = \frac{2\pi }{L}\approx 0$. This corresponds to the minimal reciprocal volume $\Delta \boldsymbol{k} = \frac{(2\pi)^3}{V}$ an oscillation takes in reciprocal space. The maximal wave vector is given by the minimal wavelength $|\boldsymbol{k}|_\textnormal{max} = \frac{^3\sqrt{N}\pi }{L }$ 
   \item The total number of modes can be calculated in reciprocal space by an integral 
   $$\sum_{k_x} \sum_{k_y} \sum_{k_z} 1\approx \left(\int_{\frac{2\pi}{L} \approx 0}^{\sqrt[3]{N} \pi/L} dk\right)^3 = \frac{N \pi^3}{V}.$$
  
  
 \item We approximate this integral of the reciprocal space by the eighth of a spherical integral using $\omega_D$ as maximal radius. This yields for the Debye angular frequency $\omega_D$:
 \begin{align*}
 \frac{N\pi^3}{V} = \frac{1}{8} 4 \pi^2 \int_0^{\omega_D} \omega^2 d\omega \\
 \omega_D = \sqrt[3]{\frac{6N \pi^2}{V}} c_S
 \end{align*}
 
\end{itemize}   

  

  Your task now is to calculate the heat capacity:
  \begin{enumerate}[resume]
  \item  Use the density for the angular frequency $D(\omega)$ to give a representation of the internal energy [Eq.~(\ref{equ:internal_energy})] as an integral. Calculate and sketch the specific heat for high and low temperatures. Use the Debye temperature $\Theta_D = \hbar \omega_D / k_B$ as reference and check if the behaviour of the specific heat is in correspondance with the physical laws.
  Compare those results to the results of the Einstein model and discuss for which temperature range which model is better suited.
  
  
  Useful hint for the approximation of the integral:
  
  Go back to exponential functions and expand the integrand in a series. Only consider leading polynoms for the approximation and use
  $$ \int_0^\infty \frac{e^x x^4}{(e^x-1)^2} dx = \frac{4}{15} \pi^4. $$
   \end{enumerate}

% \printbibliography



\vspace{2cm}
\begin{minipage}[t]{1\textwidth}
	\raggedleft
	\centering
	\includegraphics[width = 0.20\textwidth]{CC-BY_icon}
	\vspace{0.2cm}
	
	\centering
	{\Large Gerhard Dorn} \\
	https://creativecommons.org/licenses/by/4.0/legalcode
\end{minipage}



  \end{document}
  

  
