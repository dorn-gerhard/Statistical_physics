\documentclass[12pt,a4paper]{article} %scrbook}%bookof}



% makes nice font


%fills A4 page
\usepackage{a4wide}
\usepackage[inner=2cm,outer=2cm, bottom = 2.0cm]{geometry}   %nice setting for twoside binding



%\usepackage[utf8]{inputenc}   % damit man Umlaute direkt eingeben kann und diese erkannt werden.
\usepackage[T1]{fontenc}        % Umlaute werden als eine Einheit angesehen -> richtige Trennung;
% ausserdem: die T1-Fonts gibt es auch in
% größeren Größen
%\usepackage{ae,aecompl}
\usepackage{units}
\usepackage[english]{babel}      %hiermit erhält man z.B. 'Abb.' statt 'Fig.' bei Bildbeschriftungen
\usepackage[utf8]{inputenc}
%\makeglossary                   % zwischendurch makeindex disseration aufrufen
\usepackage[shortlabels]{enumitem}
\usepackage{latexsym}


%\clearscrplain
%\clearscrheadings

\usepackage[headsepline]{scrlayer-scrpage}
%\usepackage[]{scrpage2}
%\clearplainofpairofpagestyles
%\clearscrheadings
\pagestyle{scrheadings}

\cofoot[\pagemark]{\pagemark}% optionales Argument --> plain
%\clearscrheadings
\rohead{NAWI Graz}
\chead{
	Exercise Statistical Physics \\{\small Contact: gerhard.dorn@tugraz.at}}
\lohead{ITP}
\usepackage{titlesec}
\titleformat{\section}
{\normalfont\large\bfseries}{\thesection}{1em}{}


\usepackage{amsmath,amsthm,amsfonts}
%\usepackage{showkeys}
%\usepackage{epsfig}
%\usepackage{epic}
\usepackage{xr}

\usepackage{graphicx}
%\usepackage{inmthesis}
\usepackage{hyperref}
\usepackage{verbatim}
\usepackage{tikz}
\usepackage{enumitem}

\usepackage[sorting=none, hyperref=true, backref=true]{biblatex}
\addbibresource{./library.bib}
%boxes
\usepackage{mdframed}
\usepackage{tensor}
%unity operator
%\usepackage{bbold}

\begin{document}

 %\begin{center}
 %Technische Universität Graz \\
 %Institut für Theoretische Physik - computational physics \\
 %Prof. Dr. Wolfgang von der Linden \\
 %Kontakt: gerhard.dorn@tugraz.at
 %\end{center}
 
 
 \vspace{1cm}\begin{center}
{\Large \bf Exam 1, Editing time: 90 minutes}
\end{center}

\section{Dogflea experiment by Ehrenfest [4P]}
There are two dogs Arko and Bello with a total number $N = 50$ of fleas, which randomly hop from one dog to the other. In the beginning Bello is free from fleas whereas all fleas are on Arko.
\begin{enumerate}[a)] 
\item Sketch the probability distribution (diagram) of the number of fleas for dog Bello
\begin{itemize}
 \item [i)] at the very beginning,
 \item [ii)] in an intermediate phase,
 \item[iii)] and after very long time.
\end{itemize}
Mind a proper labelling (coordinate axes, peak heights and scaling)!!!

\item State the partition sum.
\item At which moment the entropy is maximal? \\
(at the beginning | in the intermediate thermalization phase | in the end)
\item At which moment the probability to find most fleas on ... is maximal
\begin{itemize}
 \item [i)] Bello (initially flea-frea)
 \item [ii)] Arko (initially full of fleas)?
\end{itemize}
(at the beginning | in the intermediate thermalization phase | in the end)

\item What is the probability to find 20 fleas on one of the two dogs after very long time?\\
\item Sketch the entropy over time.
\end{enumerate}
\section{Ideal gas in a box [4P]}
Given is an ideal gas with $N$ particles of mass $m$ in a box with volume $V$ at the temperature $T$ in a thermal equilibrium.


Which of the following parameters have to be increased (+), decreased (-) or have not to be changed at all~(0) to effect the following changes:
\begin{enumerate}[a)]
 \item The number of total collisions of particles with the wall of the box per time and area unit shall \textbf{increase}.
 \item The average energy per particle shall {\bf decrease}.
 \item The average distance between particles shall {\bf increase}.
 \item The average particle velocity in the box shall {\bf increase}.
\end{enumerate}


\begin{table}[h]
\begin{center}
 \begin{tabular}{|l| c|c|c|c|}
 \hline 
  Zustandsänderung: & $\quad V \quad$ & $\quad T \quad $ & $\quad m \quad$ & $\quad N \quad$ \\
  \hline \hline
  a) + collisions paricle with wall & &&& \\ \hline
  b) -- of $\langle E \rangle$ per particle & &&& \\ \hline
  c) + $\langle$distance$\rangle$ between particles & &&& \\ \hline
  d) + $\langle$velocity$\rangle$ & &&& \\ \hline
 \end{tabular}
\end{center}
\end{table}
 
\section{True or false [3P]}


Which of the following statements are correct. \textbf{Correct the false statements!}
% Geben Sie an, welche der folgenden Aussagen über das ideale Gas korrekt sind. Berichtigen Sie jene Aussagen, die nicht stimmen:
\begin{enumerate}[a)]
  \item The average energy of a single particle in an ideal gas is proportional to the square root of the particle mass (all particles have the same mass).
%  Die Durchschnittsenergie von Gasteilchen ist proportional zur Wurzel der Teilchenmasse.
  \item In an ideal gas, particles do not interact (attraction, repulsion).
%   Die Teilchen üben keine Kraft aufeinander aus.
  \item At a given temperature, all particles of the ideal gas have the same kinetic energy.
%  Alle Gasteilchen haben bei einer gegebenen Temperatur die gleiche kinetische Energie.
%  \item[h)] Volume of the particles, compared  to the whole volume occupied by tha gas is negligible. 
%  Das Volumen der Gasteilchen ist im Vergleich zum Gesamtvolumen, in dem sich das Gas befindet, vernachlässigbar.
  \item In an ideal gas, the average particle velocity is smaller than the most probable velocity and larger than the square root of the average of the squared velocity.
% Die mittlere Teilchengeschwindigkeit ist kleiner als die häufigste Teilchengeschwindigkeit $v_\textrm{max}$, und größer als die Wurzel des quadratischen Geschwindigkeitsmittels: \\
  $ \sqrt{\langle v^2\rangle} < \langle v \rangle < v_\textrm{max} $
  \item Effusion is a reversible process.
  \item Ergodicity of a system is a time independent feature of the system.
  \item An ideal gas in a box of volume $V$ with inelastic collisions is an ergodic system for the configuration space consisting of the surface of the energy sphere and the spatial coordinates of the box.
  %  \item Given is the grandcanonical partition sum. How do you calculate the expectation value $\langle A \rangle$ of the quantity $A$?
%  


\end{enumerate}



\section{Zipperlike chemical compound [4P]}
Given is a zipper like chemical compound consisting of $N \gg 1$ bonds (see Fig.~\ref{fig:zipper}) which is in contact with a large thermal bath of temperature $T$.
\begin{figure}[h]
\begin{center}
 %\includegraphics[width = 0.5\textwidth]{} %TO BE DRAWN
 \caption{Sketch of the zipper--like chemical compound consisting of $N = 18$ bonds, of which seven bonds are open (broken).} \label{fig:zipper}
 \end{center}
\end{figure}
The right hand side of the chemical compound is fixed to a wall. Each of the $N$ bonds can have two states: open (broken) and closed. All bonds on the left hand side of an open bond are also open (zipper like system). Thus, a configuration with closed bonds on the left side of open bonds are not allowed.
The opening (breaking) of a bond costs the energy $\Delta > 0$.

\begin{enumerate}[a)]
 \item Calculate the partition sum. Assume the following: $N\rightarrow \infty$.
 \item Calculate the internal energy $U$. 
 \item Calculate the average number of open bondings.
 \item Calculate the heat capacity. (You can combine the exponential terms to a hyperbolic function).
 \item Which behaviour does the heat capacity show for high temperatures ($T \gg 1$)?\\
Which theorem makes statements about the heat capacity at high temperatures? Show whether the behaviour is in agreement with the theorem or not.
 \item Which behaviour does the heat capacity show for low temperatures ($T \rightarrow 0$)?\\
Which theorem makes statements about the heat capacity at high temperatures? Show whether the behaviour is in agreement with the theorem or not.
 \item[g)] Calculate the entropy of the system.
\end{enumerate}


% \printbibliography



\vspace{2cm}
\begin{minipage}[t]{1\textwidth}
	\raggedleft
	\centering
	\includegraphics[width = 0.20\textwidth]{CC-BY_icon}
	\vspace{0.2cm}
	
	\centering
	{\Large Gerhard Dorn} \\
	https://creativecommons.org/licenses/by/4.0/legalcode
\end{minipage}




\end{document}

